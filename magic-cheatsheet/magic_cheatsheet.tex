\documentclass[10pt,a4paper]{article}
\usepackage[utf8]{inputenc}
\usepackage[T1]{fontenc}
\usepackage{amsmath}
\usepackage{amsfonts}
\usepackage{amssymb}
\usepackage{graphicx}
\usepackage{booktabs}
\usepackage{hyperref}
\usepackage{parskip}
\author{Harald Pretl\\Institute for Integrated Circuits, Johannes Kepler University Linz}
\title{Magic VLSI Cheatsheet}
\date{\copyright~2021}
\newcommand{\key}[1]{\texttt{\textbf{#1}}}
\newcommand{\mac}[1]{\texttt{#1}}
\newcommand{\argu}[1]{\textit{#1}}
\newcommand{\mouse}[1]{\texttt{#1}}
\begin{document}
	\maketitle
	
	This cheat sheet compiles useful commands (to be entered in the Tcl window) and bind keys (in the layout window) as well as a summary of the mouse operations for the different tool modes. This compilation relates to Magic 8.3 (see \url{https://github.com/RTimothyEdwards/magic}).
	
	\newpage

	\begin{tabular}{p{0.3\textwidth}p{0.7\textwidth}}
		\toprule
		\multicolumn{2}{l}{Useful commands:}\\
		\midrule
		\mac{addpath} [\argu{path}] & add search \argu{path} of cells\\
		\mac{array} \argu{xsize} \argu{ysize} & array the selection \argu{xsize}-times in X, \argu{ysize}-times in Y, spacing by box\\
		\mac{contact} \argu{type} & create contact inside box of intersecting layers\\
		\mac{corner} \argu{dir1} \argu{dir2} & draws a corner inside box, first in \argu{dir1}, then in \argu{dir2} (cf.~\mac{fill})\\
		\mac{drc} \mac{find} & locate DRC error\\
		\mac{drc} \mac{why} & explain DRC errors in box\\
		\mac{dump} \argu{cell} & copy given \argu{cell} into current layout cell\\
		\mac{erase} \argu{layers} & erase comma-separated list of \argu{layers} in box\\
		\mac{erase} \mac{*} & all layers but not labels\\
		\mac{erase} \mac{\$} & erase everything (like \key{CTRL-D})\\
		\mac{erase labels} & erase selected label(s) in box\\
		%\mac{exit} & exits magic\\ don't use exit, as it might cause trouble
		\mac{fill} \argu{dir} & extend layers from one side of box to the other in the given \argu{dir}ection (cf.~\mac{corner})\\
		\mac{findlabel} \argu{label} & center the box on the \argu{label}\\
		\mac{flatten} \argu{cell} & flatten the current edit cell incl. hierarchy into a new \argu{cell}\\
		\mac{flush} & revert edit cell to last state on disk\\
		\mac{get} \argu{cell} & instantiate \argu{cell} into current edit cell\\
		\mac{help} [\argu{pattern}] & shows help for various commands\\
		\mac{identify} \argu{cell-id} & name the selected cell as \argu{cell-id}\\
		\mac{iroute} & interactive point-to-point router, from cursor to box\\
		\mac{label} \argu{string} [\argu{position} [\argu{layer}]] & create a point, line or box label\\
		\mac{load} \argu{cell} & open the \argu{cell} in the layout window\\
		\mac{macro} [\argu{window\_type}] [\argu{key} [\argu{value}]] & new macro definition \\
		\mac{measure} [\mac{hor}|\mac{ver}|\mac{auto}] & create a ruler in the box\\
		\mac{paint} \argu{layers} & paint the comma-separated list of \argu{layers} in box\\
		\mac{plow} \argu{dir} & move layout structures incl. connectivity in \argu{dir}ection, see manual\\
		\mac{port make} & create a port from selected label\\
		\mac{port index} \argu{number} & assign a port index (must match the number from schematic)\\ 
		\mac{quit} \mac{[-nocheck]} & exit magic (using \mac{-nocheck} discards unsaved changes)\\
		\mac{see} \mac{[no]} \argu{layers} & show or hide comma-separated list of \argu{layers}\\
		\mac{see} \mac{[no]} \mac{*} & show or hide all layers\\
		\mac{select save} \argu{file} & Save selection as new \argu{file}.mag\\
		\mac{straighten} \argu{dir} & try to repair jogs by pulling in \argu{dir}ection\\
		\mac{tool} & changes the currently used \mouse{tool} to (box, wiring, netlist, pick)\\
		\mac{redraw} & redraws the window\\
		\mac{unmeasure} & remove all rulers in box\\
		\mac{what} & print information about the current selection\\
		\mac{xload} \argu{cell} & open the \argu{cell} in the layout window unexpanded\\
		\mac{xview} & full view, with unexpanded cells\\
		\bottomrule
	\end{tabular}
	
	\newpage

	\begin{tabular}{p{0.3\textwidth}p{0.7\textwidth}}
		\toprule
		\multicolumn{2}{l}{Bind keys:}\\
		\midrule
		\key{a} & select everything in box\\
		\key{A} & select additional objects in box\\
		\key{CTRL-A} & deselect objects in box\\
		\key{b} & print information about the \mac{box}\\
		\key{B} & center window on the box\\
		\key{c} & \mac{copy}\\
		\key{d} & \mac{delete} selection\\
		\key{CTRL-D} & delete all layers and labels in box selected by cursor\\
		\key{e} & makes the selected cell the new \mac{edit} cell\\
		\key{F} & flip the selection upside-down\\
		\key{g} & toggle \mac{grid}\\
		\key{G} & toggle \mac{grid 2}\\
		\key{i} & select instance under cursor\\
		\key{I} & select additional instance under cursor\\
		\key{CTRL-I} & deselect instance under cursor\\
		\key{m} & \mac{move} selection to cursor position\\
		\key{o} & \mac{openwindow} (with current selection)\\
		\key{r} & \mac{rotate} 90\\
		\key{R} & \mac{rotate} -90\\
		\key{s} & \mac{select} objects under cursor\\
		\key{S} & \mac{select} additional objects under cursor\\
		\key{CTRL-S} & deselect objects under cursor\\
		\key{u} & \mac{undo} the last operation\\
		\key{U} & \mac{redo} the last undo\\
		\key{v} & \mac{view}, full view of layout\\
		\key{w} & \mac{writeall} changed cells to disk\\
		\key{x} & \mac{expand} cells (make visible) in box\\
		\key{X} & \mac{unexpand} cells (hide) inside the box\\
		\key{CTRL-X} & toggle the visibility of cells in box\\
		\key{y} & \mac{drc why}\\
		\key{z} & zoom out\\
		\key{Z} & zoom in\\
		\key{CTRL-Z} & zoom in on box\\
		\key{>} & edit the selected cell in layout window (cf.~\key{<})\\
		\key{<} & return from editing a cell (cf.~\key{>})\\
		\key{,} & clear selection\\
		\key{:} & directly jump to command window (like VI)\\
		\key{SPACE} & change the tool\\
		\key{SHIFT + SPACE} & change to the box tool\\
		\key{KEYPAD-2} & move selection south, equivalent to `\mac{move} s 1`\\
		\key{SHIFT + KEYPAD-2} & stretch selection south, equivalent to `\mac{stretch} s 1`\\
		\key{KEYPAD-4} & move selection west, equivalent to `\mac{move} w 1`\\
		\key{SHIFT + KEYPAD-4} & stretch selection west, equivalent to `\mac{stretch} w 1`\\
		\key{KEYPAD-6} & move selection east, equivalent to `\mac{move} e 1` \\
		\key{SHIFT + KEYPAD-6} & stretch selection east, equivalent to `\mac{stretch} e 1`\\
		\key{KEYPAD-8} & move selection north, equivalent to `\mac{move} n 1` \\
		\key{SHIFT + KEYPAD-8} & stretch selection north, equivalent to `\mac{stretch} n 1`\\
		\bottomrule
	\end{tabular}

	\newpage
	
        \begin{tabular}{p{0.3\textwidth}p{0.7\textwidth}}
                \toprule
		\multicolumn{2}{l}{Bind keys (IIC add-on):}\\
                \midrule
                \key{f} & full layout view\\
                \key{CTRL-F} & flip the selection sideways\\
                \key{k} & measure (ruler)\\
                \key{K} & remove rulers\\
                \key{p} & draw a wiring path\\
                \key{q} & open poperty dialog of p-cell\\
                \key{?} & show what is contained in box\\
                \key{0} & show locali\\
                \key{!} & show metal-1\\
                \key{@} & show metal-2\\
                \key{\#} & show metal-3\\
                \key{\$} & show metal-4\\
                \key{\%} & show metal-5\\
                \key{9} & show all layers\\
                \bottomrule
        \end{tabular}

	\newpage

	\begin{tabular}{p{0.3\textwidth}p{0.7\textwidth}}
		\toprule
		\multicolumn{2}{l}{Mouse operations:}\\
		\midrule
		\multicolumn{2}{l}{Box tool:}\\
		\mouse{LEFT} & set lower-left corner of box\\
		\mouse{RIGHT} & set upper-right corner of box\\
		\mouse{MIDDLE} & \mac{paint} layer pointed with cursor in box\\
		\mouse{SHIFT + MIDDLE} & \mac{erase} layer pointed with cursor in box\\
		\midrule
		\multicolumn{2}{l}{Layer selection tool:}\\
		\mouse{LEFT} & show selected layer, cf.~\mac{see}\\
		\mouse{RIGHT} & hide selected layer, cf.~\mac{see} no\\
		\midrule
		\multicolumn{2}{l}{Wiring tool:}\\
		\mouse{LEFT} & pick wire material and size from cursor\\
		\mouse{MIDDLE} & place wire at cursor\\
		\mouse{RIGHT} & cancel wire\\
		\mouse{SHIFT + LEFT} & change to higher layer (e.g. m1 to m2)\\
		\mouse{SHIFT + RIGHT} & change to lower layer (e.g. m2 to m1)\\
		\mouse{SHIFT + MIDDLE} & place a contact at cursor and continue on higher layer\\
		\midrule
		\multicolumn{2}{l}{Pick tool:}\\
		\mouse{LEFT} & remove current selection and follow cursor\\
		\mouse{MIDDLE} & place a copy and continue\\
		\mouse{RIGHT} & cancel\\
		\mouse{SHIFT + MIDDLE} & make a copy of selection and follow cursor\\
		\bottomrule
	\end{tabular}
	
\end{document}
